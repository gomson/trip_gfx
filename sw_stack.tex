\chapter{The Software Stack}
\label{ch:swstack}
\bibliographystyle{plainnat}

This section assumes a reasonably recent Windows version (Vista or later),
which use the WDDM~\citep{wddm} driver model. Older driver models (and other
platforms) are somewhat different, but that's outside the scope of this
text---I just picked WDDM because it's probably the most relevant model on
PCs right now.

\section{Application and API}

It all starts with the application. On PC, all communication between an app and
the GPU is mediated by the graphics API; apps may occasionally get direct
access to memory that's GPU-addressable (such as Vertex Buffers or Textures),
but on PC they can't directly generate native GPU commands \footnote{Not
officially, anyway; since the UMD writes command buffers and runs in user mode,
an app could conceivably figure out where the UMD stores its current write
pointer and insert GPU commands manually, but that's not exactly supported or
recommended behavior.} --- all that has to go through the API and the driver.

The API is the recipient of the app's resource creation, state-setting, and draw
calls. The API runtime keeps track of the current state your app has set,
validates parameters and does other error and consistency checking, manages
user-visible resources, and may or may not validate shader code and shader
linkage (it does in D3D, while in OpenGL this is handled at the driver level).
It can also merge batches if possible. It then packages it all up nicely and
hands it over to the graphics driver---more precisely, the user-mode driver.

\section{The User-Mode Driver (UMD)}

This is where most of the ``magic'' on the CPU side happens. If your app
crashes because of some API call you did, it will usually be in here :). It's
called ``nvd3dum.dll'' (NVidia) or ``atiumd*.dll'' (AMD). As the name suggests,
this is user-mode code; it's running in the same context and address space as
your app (and the API runtime) and has no elevated privileges whatsoever. It
implements a lower-level API (the DDI~\citep{umd-ddi}) that is called by D3D;
this API is fairly similar to the one you're seeing on the surface, but a bit
more explicit about things like memory and state management.

This module is where things like shader compilation happen. D3D passes a
pre-validated shader token stream to the UMD---i.e. it's already checked that
the code is valid in the sense of being syntactically correct and obeying D3D
constraints (using the right types, not using more textures/samplers than
available, not exceeding the number of available constant buffers, stuff like
that). This is compiled from HLSL code and usually has quite a number of
high-level optimizations (various loop optimizations, dead-code elimination,
constant propagation, predicating ifs etc.) applied to it---this is good news
since it means the driver benefits from all these relatively costly
optimizations that have been performed at compile time. However, it also has a
bunch of lower-level optimizations (such as register allocation and loop
unrolling) applied that drivers would rather do themselves; long story short,
this usually just gets immediately turned into a intermediate representation
(IR) and then compiled some more; shader hardware is close enough to D3D
bytecode that compilation doesn't need to work wonders to give good results
(and the HLSL compiler having done some of the high-yield and high-cost
optimizations already definitely helps), but there's still lots of low-level
details (such as HW resource limits and scheduling constraints) that D3D
neither knows nor cares about, so this is not a trivial process.

And of course, if your app is a well-known game, programmers at NV/AMD have
probably looked at your shaders and wrote hand-optimized replacements for their
hardware - though they better produce the same results lest there be a scandal
:). These shaders get detected and substituted by the UMD too. You're welcome.

More fun: Some of the API state may actually end up being compiled into the
shader - to give an example, relatively exotic (or at least infrequently used)
features such as texture borders are probably not implemented in the texture
sampler, but emulated with extra code in the shader (or just not supported at
all). This means that there's sometimes multiple versions of the same shader
floating around, for different combinations of API states.

Incidentally, this is also the reason why you'll often see a delay the first
time you use a new shader or resource; a lot of the creation/compilation work
is deferred by the driver and only executed when it's actually necessary (you
wouldn't believe how much unused crap some apps create!). Graphics programmers
know the other side of the story - if you want to make sure something is
actually created (as opposed to just having memory reserved), you need to issue
a dummy draw call that uses it to "warm it up". Ugly and annoying, but this has
been the case since I first started using 3D hardware in 1999 - meaning, it's
pretty much a fact of life by this point, so get used to it. :)

Anyway, moving on. The UMD also gets to deal with fun stuff like all the D3D9
``legacy'' shader versions and the fixed function pipeline - yes, all of that
will get faithfully passed through by D3D. The 3.0 shader profile ain't that
bad (it's quite reasonable in fact), but 2.0 is crufty and the various 1.x
shader versions are seriously weird---remember 1.3 pixel shaders? Or, for that
matter, the fixed-function vertex pipeline with vertex lighting and such? Yeah,
support for all that's still there in D3D and the guts of every modern graphics
driver, though of course they just translate it to newer shader versions by now
(and have been doing so for quite some time).

Then there's things like memory management. The UMD will get things like
texture creation commands and need to provide space for them. Actually, the UMD
just suballocates some larger memory blocks it gets from the KMD (Kernel-Mode
Driver); actually mapping and unmapping pages (and managing which part of video
memory the UMD can see, and conversely which parts of system memory the GPU may
access) is a kernel-mode privilege and can't be done by the UMD.

But the UMD can swizzle textures, for example - that is, go from linear pixel
layout to something that's more likely to get good cache hit rates during 3D
rendering; we'll see this later. Some GPUs can also do the swizzling
themselves, during a 2D blit or copy. The UMD can also schedule transfers
between system memory and (mapped) video memory and the like. Most importantly,
it can also write command buffers (or ``DMA buffers'' - I'll be using these two
names interchangeably) once the KMD has allocated them and handed them over. A
command buffer contains, well, commands :). All your state-changing and drawing
operations will be converted by the UMD into commands that the hardware
understands. As will a lot of things you don't trigger manually - such as
uploading textures and shaders to video memory.

In general, drivers will try to put as much of the actual processing into the
UMD as possible; the UMD is user-mode code, so anything that runs in it doesn't
need any costly kernel-mode transitions, it can freely allocate memory, farm
work out to multiple threads, and so on - it's just a regular DLL (even though
it's loaded by the API, not directly by your app). This has advantages for
driver development too - if the UMD crashes, the app crashes with it, but not
the whole system; it can just be replaced while the system is running (it's
just a DLL!); it can be debugged with a regular debugger; and so on. So it's
not only efficient, it's also convenient.

But there's a big elephant in the room that I haven't mentioned yet.

\subsection*{Did I say ``user-mode driver''? I meant ``user-mode drivers.''}

As said, the UMD is just a DLL. Okay, one that happens to have the blessing of
D3D and a direct pipe to the KMD, but it's still a regular DLL, and in runs in
the address space of its calling process.

But we're using multi-tasking OSes nowadays. In fact, we have been for some
time.

This "GPU" thing I keep talking about? That's a shared resource. There's only
one that drives your main display (even if you use SLI/Crossfire). Yet we have
multiple apps that try to access it (and pretend they're the only ones doing
it). This doesn't just work automatically; back in The Olden Days, the solution
was to only give 3D to one app at a time, and while that app was active, all
others wouldn't have access. But that doesn't really cut it if you're trying to
have your windowing system use the GPU for rendering. Which is why you need
some component that arbitrates access to the GPU and allocates time-slices and
such.

\section{The scheduler}

This is a system component, part of the OS - note the ``the'' is somewhat
misleading; I'm talking about the graphics scheduler here, not the CPU or IO
schedulers. This does exactly what you think it does - it arbitrates access to
the 3D pipeline by time-slicing it between different apps that want to use it.
A context switch incurs, at the very least, some state switching on the GPU
(which generates extra commands for the command buffer) and possibly also
swapping some resources in and out of video memory.  And of course only one
process gets to actually submit commands to the 3D pipe at any given time.

You'll often find console programmers complaining about the fairly high-level,
hands-off nature of PC 3D APIs, and the performance cost this incurs. But the
thing is that 3D APIs/drivers on PC really have a more complex problem to solve
than console games - they really do need to keep track of the full current
state for example, since someone may pull the metaphorical rug from under them
at any moment! They also work around broken apps and try to fix performance
problems behind their backs; this is a rather annoying practice that no-one's
happy with, certainly including the driver authors themselves, but the fact is
that the business perspective wins here; people expect stuff that runs to
continue running (and doing so smoothly). You just won't win any friends by
yelling ``BUT IT'S WRONG!'' at the app and then sulking and going through an
ultra-slow path.

Anyway, on with the pipeline. Next stop: Kernel mode!
 
\section{The Kernel-Mode Driver (KMD)}

This is the part that actually deals with the hardware. There may be multiple
UMD instances running at any one time, but there's only ever one KMD, and if
that crashes, then boom you're dead - used to be "blue screen" dead, but by now
Windows actually knows how to kill a crashed driver and reload it (progress!).
As long as it happens to be just a crash and not some kernel memory corruption
at least - if that happens, all bets are off.

The KMD deals with all the things that are just there once. There's only one
GPU memory, even though there's multiple apps fighting over it. Someone needs
to call the shots and actually allocate (and map) physical memory. Similarly,
someone must initialize the GPU at startup, set display modes (and get mode
information from displays), manage the hardware mouse cursor, program the HW
watchdog timer so the GPU gets reset if it stays unresponsive for a certain
time, respond to interrupts, and so on. This is what the KMD does.

There's also this whole content protection/DRM bit about setting up a
protected/DRM'ed path between a video player and the GPU so no the actual
precious decoded video pixels aren't visible to any dirty user-mode code that
might do awful forbidden things like dump them to disk (...whatever). The KMD
has some involvement in that too.

Most importantly for us, the KMD manages the \emph{actual} command buffer.  You
know, the one that the hardware actually consumes. The command buffers that the
UMD produces aren't the real deal --- as a matter of fact, they're just random
slices of GPU-addressable memory. What actually happens with them is that the
UMD finishes them, submits them to the scheduler, which then waits until that
process is up and then passes the UMD command buffer on to the KMD.  The KMD
then writes a call to command buffer into the main command buffer, and
depending on whether the GPU command processor can read from main memory or
not, it may also need to DMA it to video memory first. The main command buffer
is usually a (quite small) ring buffer --- the only thing that ever gets
written there is system/initialization commands and calls to the ``real'',
meaty 3D command buffers.

But this is still just a buffer in memory right now. Its position is known to
the graphics card --- there's usually a read pointer, which is where the GPU is
in the main command buffer, and a write pointer, which is how far the KMD has
written the buffer yet (or more precisely, how far it has \emph{told} the GPU
it has written yet). These are hardware registers, and they are memory-mapped
--- the KMD updates them periodically, usually whenever it submits a new chunk
of work. How memory transfers to and from the GPU work will be explained in
chapter~\ref{ch:memory}.

\section{Aside: OpenGL and other platforms}

OpenGL is fairly similar to what I just described, except there's not as sharp
a distinction between the API and UMD layer. And unlike D3D, the (GLSL) shader
compilation is not handled by the API at all, it's all done by the driver. An
unfortunate side effect is that there are as many GLSL frontends as there are
3D hardware vendors, all of them basically implementing the same spec, but with
their own bugs and idiosyncrasies. Not fun. And it also means that the drivers
have to do all the optimizations themselves whenever they get to see the
shaders - including expensive optimizations. The D3D bytecode format is really
a cleaner solution for this problem - there's only one compiler (so no slightly
incompatible dialects between different vendors!) and it allows for some
costlier data-flow analysis than you would normally do.

Open Source implementations of GL tend to use either Mesa or Gallium3D, both of
which have a single shared GLSL frontend that generates a device-independent IR
and supports multiple pluggable backends for actual hardware. In other words,
that space is fairly similar to the D3D model.

\section{Further Reading}

This is just a very coarse overview of the WDDM graphics stack, meant to give
you a general idea of what fits where. For details, refer to the official WDDM
documentation~\citep{wddm}.

\bibliography{sw_stack}{}
