\chapter{GPU Memory Subsystem and Host Interface}
\label{ch:memory}
\bibliographystyle{plainnat}

Before a GPU can do anything with the DMA buffers written by the (kernel mode) driver,
it needs to be able to read them first, which entails communication between CPU and GPU
and interfacing with GPU memory. On the CPU side, processors have long since stopped
passing every memory access to RAM chips, and instead employ a sophisticated memory
hierarchy with multiple levels of caches.\footnote{Caches of various kinds are by far
the biggest part of contemporary CPUs in terms of area.} GPU memory systems have very
different goals and operate under different design constraints; while they still use
caches, they generally tend to be smaller and favor different usage patterns. This unique
type of memory architecture greatly influences the way shader cores handle loads and stores,
and resonates through the whole design of modern GPUs.

This chapter will also briefly cover the host interface, the part of the GPU that talks
to the outside world (usually the CPU).

\section{The memory subsystem}

As a regular programmer, the most important thing to understand about the memory
subsystem of GPUs is that, compared to a CPU, it is both incredibly fast and incredibly
slow at the same time. Fast, in that GPU memory systems deliver incredibly amounts of
\emph{bandwidth}: At the time of writing, the fastest desktop CPU you can buy from Intel,
the Core i7-3960X, has a theoretical peak memory bandwidth of 51.2~GB/s~\citep{membw-i7-3960x};
a more pedestrian (but still very much high-end) Core i7-2700K maxes out at 21~GB/s~\citep{membw-i7-2700k}.
For comparison, recent high-end discrete GPUs from AND and Nvidia boast memory bandwidths of
264~GB/s~\citep{membw-radeonhd-7970} and 192~GB/s~\citep{membw-geforce-gtx680}, respectively---a
tremendous difference from the values we see for CPUs.

At the same time, GPU memory subsytems are very slow, in the sense that they have high
memory access latency. Again some numbers: memory access latency on Core~i7 level CPUs comes
out at about 45ns~\citep{memlat-i7}, while Nvidia quotes about 400--800 cycles of memory access
latency for Fermi generation chips~\citep{memlat-nvidia}. At the 1544MHz high-end Fermi parts
typically run at, this means about 260--520ns---again a 5$\times$ difference between high-end
CPUs and GPUs, but this time in the opposite direction.

This is not a coincidence: GPUs want high memory bandwidth and are willing to trade a significant
increase in latency (and also power draw, which I will ignore in this text) in return. This is part
of a general pattern: GPUs greatly favor throughput over latency, so when there is an opportunity to
trade one for the other, GPUs will go for throughput every time. Of course, this is easier said than
done: optimizing for throughput works differently than optimizing for latency does, and software
developers tend to be far more familiar with the latter. We will see this subject come up again and again
throughout this text.

But first things first: how do we get high memory bandwidth in the first place? There's several ways.
The first and conceptually easiest is to just increase memory clock rate. All other things being equal,
if we increase the number of memory transfers per second, we will manage to access more memory in a second.
However, power draw goes up dramatically with increased clock frequencies, which limits practical clock rates;
as of this writing, mainstream GPUs ship with memory clock rates of 1.5GHz or less.\footnote{GPU vendors
sometimes like to quote memory \emph{data rate} instead of \emph{clock frequency}; for GDDR5, data rate is $4 \times$
the clock frequency, while for GDDR3 and DDR2/3 memory it is $2 \times$ the clock frequency.}

If we can't increase the clock any further, the next option is to throw more hardware at the problem, by
making the memory bus wider. To give a concrete example, a single GDDR5 memory module has an
IO width of 32 bits~\citep{elpida-gddr5}, meaning that a GDDR5 chip transfers data 32 bits at a time.\footnote{Strictly
speaking, GDDR5 memory also supports ``clamshell mode'' where two GDDR5 chips share the same address/command bus
and only transfer 16 bits at a time; from the memory controller's point of view, normal and clamshell mode
look basically the same, and I'll ignore the distinction here.} So if a GPU uses GDDR5 and has a 256-bit
wide memory subsystem, that means the memory controller is connected to 8 memory chips. This in turns means that
to max out memory bandwidth, there needs to be a transfer from or to each of those memory chips in every single clock
cycle.

Which leads to the third way to increase effective memory bandwidth: try not to waste any cycles! To
maximize net bandwidth, GPU memory controllers buffer and reorder memory accesses very aggressively.
Assuming that memory accesses are reasonably balanced across different memory chips, i.e. no chip is used
much more often than any other, we can get fairly close to actually getting one memory transfer per memory
chip and clock if we're willing to queue memory accesses up for a while. Memory transfers tend to be generated
in short, high-rate bursts; queuing allows us to smooth out these bumps and turn them into a fairly steady
level of memory requests that then gets serviced by the memory chips. In general, the longer the queues are,
the smoother things will go. The flipside is that longer queues automatically mean --- and this is where the
trade-off comes into play --- longer latency. CPUs try to keep queue lengths short (at least while they can,
i.e.\@ as long as the memory system isn't under high load), which gives them low latency but also costs them
effective memory bandwidth. GPUs queue very aggressively, so they are really good at keeping their memory chips
busy, but they pay for it with hundreds of cycles of extra latency.

However, this only gets us half the way towards our goal of not wasting cycles. We're doing our best to make
sure that the memory chips are busy every cycle; however, there's several ways that a DRAM chip can be busy,
and most of them don't involve any transfers of payload data, so these cycles are essentially lost as far as
bandwidth is concerned. All of these ``wasted'' cycles are caused by the internal organization of DRAM.

\section{How DRAM works}

\bibliography{memory}{}
